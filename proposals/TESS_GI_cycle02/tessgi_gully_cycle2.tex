%% LaTeX template for the science justification & technical
%% feasibility to be submitted as part of a TESS Guest Investigator
%% Program proposal. This template is based on the proposal template
%% used by the NuSTAR mission.
%%
%% TESS Guest Investigator Proposal Cycle 2 template
%% V1.0
%% 2017-08-04
%% V1.1
%% 2019-02-07

%%%%%%%%%%%%%%%%%%%%%%%%%%%
%%%%% DOCUMENT FORMAT %%%%%
%%%%%%%%%%%%%%%%%%%%%%%%%%%

%% The default font was chosen to be easily readable while allowing
%% sufficient material to be included.

%% Please note that the proposal will be printed on US Letter size paper,
%% 8.5 in x 11 in, and that formatting the text for other sizes will
%% generally cause layout problems and may result in text being cut
%% off near the edges. PLEASE DO NOT CHANGE THE 'LETTERPAPER' OPTION
%% IN THE DOCUMENTCLASS COMMAND.

%%%%%%%%%%%%%%%%%%%%%%%%%%%%%%%%%%%%%%%%%%%%%%
%%%%% Default format: 11pt single column %%%%%
%%%%%%%%%%%%%%%%%%%%%%%%%%%%%%%%%%%%%%%%%%%%%%

%% NOTE: NASA ROSES requires body font size to be no smaller than 15
%% characters per inch (equivalent to Times Roman 12 point).
%%
%% Minimum margin size is 1 inch from top, bottom, and sides.

\documentclass[letterpaper,11pt]{article}

%%%%%%%%%%%%%%%%%%%%%%%%%%%%%%%%%%
%%%%% HOW TO INCLUDE FIGURES %%%%%
%%%%%%%%%%%%%%%%%%%%%%%%%%%%%%%%%%

%% Please see the ``Included packages'' section below.

%%%%%%%%%%%%%%%%%%%%%%%%%%%%%
%%%%% Included packages %%%%%
%%%%%%%%%%%%%%%%%%%%%%%%%%%%%

\usepackage{graphics,graphicx}
\usepackage{aas_macros}
\usepackage[comma,authoryear]{natbib}
\bibpunct{(}{)}{;}{a}{}{;}

\usepackage{caption}
\usepackage{subcaption}


\usepackage[backref,breaklinks,colorlinks,urlcolor=blue,citecolor=blue,linkcolor=blue]{hyperref}
\bibliographystyle{plainnat}
%\usepackage{cleveref}
\usepackage{enumerate}
\usepackage{amsmath,amssymb}
\usepackage{bm}
\usepackage{color}
\usepackage[utf8]{inputenc}

\newcommand{\prob}{{\rm prob}}
\newcommand{\qN}{\{q_i\}_{i=1}^N}
\newcommand{\qM}{\{q_{im}\}_{i=1,m=0}^{N,M}}
\newcommand{\yN}{\{y_i\}_{i=1}^N}

\newcommand{\kms}{ \textrm{km s}^{-1} }

\newcommand{\vM}{\mathsf{M}}
\newcommand{\vD}{\mathsf{D}}
\newcommand{\vR}{\mathsf{R}}
\newcommand{\vC}{\mathsf{C}}
\newcommand{\fM}{ \vec{{\bm M}}}
\newcommand{\fMi}{M_i}
\newcommand{\fD}{ \vec{{\bm D}}}
\newcommand{\fDi}{D_i}
\newcommand{\fR}{ {\bm R}}
\newcommand{\dd}{\,{\rm d}}
\newcommand{\trans}{\mathsf{T}}
\newcommand{\teff}{T_\textrm{eff}}
\newcommand{\logg}{\log g}
\newcommand{\Z}{[{\rm Fe}/{\rm H}]}
\newcommand{\A}{[\alpha/{\rm Fe}]}
\newcommand{\vsini}{v \sin i}
\newcommand{\matern}{Mat\'{e}rn}
\newcommand{\HK}{$\textrm{H}_2$O-K2}
\newcommand{\cc}[2]{c_{#2}^{(#1)}} 

\newcommand{\flam}{f_\lambda}
\newcommand{\vt}{ {\bm \theta}}
\newcommand{\vT}{ {\bm \Theta}}
\newcommand{\vp}{ {\bm \phi}}
\newcommand{\vP}{ {\bm \Phi}}
\newcommand{\cheb}{ \vp_{\mathsf{P}}}
\newcommand{\chebi}[1]{ \vp_{\textrm{Cheb}_{#1}}}
\newcommand{\Cheb}{ \vP_{\textrm{Cheb}}}
\newcommand{\Chebi}[1]{ \vP_{\textrm{Cheb}_{\ne #1}}} 
\newcommand{\cov}{ \vp_{\mathsf{C}}}
\newcommand{\covi}[1]{ \vp_{\textrm{cov}_{#1}}} 
\newcommand{\Cov}{ \vP_{\textrm{cov}}}
\newcommand{\Covi}[1]{ \vP_{\textrm{cov}_{\ne #1}}} 

\newcommand{\allParameters}{\vT} 
\newcommand{\nuisanceParameters}{\vP} 

\newcommand{\KK}{\mathcal{K}}
\newcommand{\Kglobal}{\KK^{\textrm{G}}}
\newcommand{\Klocal}{\KK^{\textrm{L}}}

\newcommand{\Gl}{Gl\,51}
\newcommand{\PHOENIX}{{\sc Phoenix}}

% Appendix commands
\newcommand{\wg}{\mathbf{w}^\textrm{grid}}
\newcommand{\wgh}{\hat{\mathbf{w}}^\textrm{grid}}

\newcommand{\Sg}{\mathbf{\Sigma}^\textrm{grid}}


\newcommand{\todo}[1]{ \textcolor{blue}{\\TODO: #1}}
\newcommand{\comm}[1]{ \textcolor{red}{SA: #1}}
\newcommand{\hili}[1]{ \textcolor{green}{#1}}
\newcommand{\ctext}[1]{ \textcolor{blue}{\% #1}}

%% Feel free to modify the included packages list to use your
%% favorite packages.

%% In the graphics and graphicx packages, Postscript and eps figures
%% can be included using the \includegraphics command. The graphics
%% package is part of standard LaTeX2e and provides a basic way of including a
%% figure. The graphicx package is not standard, but extends the
%% \includegraphics command to make it more user-friendly. If graphicx
%% is not available on your system please remove it from the list of
%% included packages above.

%% Syntax:
%% In the graphics package:
%%
%% \begin{figure}
%% \includegraphics[llx,lly][urx,ury]{file}
%% \end{figure}
%%
%% where ll denotes 'lower left' and ur 'upper right' and the x and y
%% values are the coordinates of the PostScript bounding box in
%% points. There are 72 points in an inch.
%%
%% In the graphicx package:
%%
%% \begin{figure}
%% \includegraphics[key=val,key=val,...]{file}
%% \end{figure}
%%
%% where some of the useful keys are: angle, width, height,
%% keepaspectratio (='true' or 'false') and scale. Bounding box values
%% can be given as [bb=llx lly urx ury].
%%
%% In either case you have to use LaTeX figure placement commands to
%% position the figure on the page; \includegraphics will not do
%% that. Both these commands also have other options that are listed
%% in the LaTeX manual (for the graphics package) and in 'The LaTeX
%% Graphics Companion' (for the graphicx package).



%%%%%%%%%%%%%%%%%%%%%%%%%%%
%%%%% Page dimensions %%%%%
%%%%%%%%%%%%%%%%%%%%%%%%%%%

\setlength{\textwidth}{6.5in}
\setlength{\textheight}{9in}
\setlength{\topmargin}{-0.0625in}
\setlength{\oddsidemargin}{0in}
\setlength{\evensidemargin}{0in}
\setlength{\headheight}{0in}
\setlength{\headsep}{0in}
\setlength{\hoffset}{0in}
\setlength{\voffset}{0in}



%%%%%%%%%%%%%%%%%%%%%%%%%%%%%%%%%%
%%%%% Section heading format %%%%%
%%%%%%%%%%%%%%%%%%%%%%%%%%%%%%%%%%

\makeatletter
\renewcommand{\section}{\@startsection%
{section}{1}{0mm}{-\baselineskip}%
{0.5\baselineskip}{\normalfont\Large\bfseries}}%
\makeatother

%%%%%%%%%%%%%%%%%%%%%%%%%%%%%%%%%%%%%
%%%%% Some Useful Abbreviations %%%%%
%%%%%%%%%%%%%%%%%%%%%%%%%%%%%%%%%%%%%
\newcommand{\tess}{{\it TESS}}
\newcommand{\jwst}{{\it JWST}}
\newcommand{\kepler}{{\it Kepler}}
\newcommand{\ktwo}{{K2}}
\newcommand{\hst}{{\it HST}}
\newcommand{\msun}{$M_{\odot}$}
\newcommand{\rsun}{$R_{\odot}$}
\newcommand{\lsun}{$L_{\odot}$}
\newcommand{\re}{$R_{\oplus}$}
\newcommand{\me}{$M_{\oplus}$}
\newcommand{\rj}{$R_{\textrm{\scriptsize Jup}}$}
\newcommand{\mj}{$M_{\textrm{\scriptsize Jup}}$}
\newcommand{\ms}{m~s$^{-1}$}



%%%%%%%%%%%%%%%%%%%%%%%%%%%%%
%%%%% Start of document %%%%%
%%%%%%%%%%%%%%%%%%%%%%%%%%%%%

\begin{document}
\pagestyle{plain}
\pagenumbering{arabic}



%%%%%%%%%%%%%%%%%%%%%%%%%%%%%
%%%%% Title of proposal %%%%%
%%%%%%%%%%%%%%%%%%%%%%%%%%%%%

\begin{center}
\bfseries\uppercase{%
%%
%% ENTER TITLE OF PROPOSAL BELOW THIS LINE
Starspots confound derived stellar properties and a concept for probabilistic mitigation
%%
%%
}
\end{center}



%%%%%%%%%%%%%%%%%%%%%%%%%%%%%%%%%%%%%%%%%
%%%%% Body of science justification %%%%%
%%%%% and technical feasibility     %%%%%
%%%%%%%%%%%%%%%%%%%%%%%%%%%%%%%%%%%%%%%%%

\section{Introduction}

Summarize the problem being addressed and give an overview of how your investigation will help.
Why \tess, why now?


\section{Scientific Justification}



\subsection{Starspots as confounders of isochronal ages}
For example, see \cite{2015ApJ...807..174S} and \cite{2015arXiv150706460F}).  Specifically, stellar evolution models


%% Trigger Criteria section:
%% comment this section out on proposals not asking for
%% Target of Opportunity (ToO) observations.

\subsection{Transit Light Source Effect}
The diminution of flux arising from occultations of the stellar disk contains information about the relative sizes of the occulting bodies.  A commonsense exoplanet heuristic dictates that the relative transit depth $D$ (in units of ppm or $\%$) equals the ratio of the bodies' projected areas ($R_p^2/R_\star^2$).  Addtional correction factors account for limb darkening \citep{2002ApJ...580L.171M}.  Starspots on a stellar surface can exhibit spatial segrations\footnote{Sunspots reside on the Sun's mid-latitudes, and starspots can exist near the stellar poles, as evinced in spatially resolved interferometry of nearby giant stars \citep{2016Natur.533..217R}} that yield \emph{unocculted starspots}: spots that are present on the surface but are not traversed during event during transit.  These unocculted starspots confound the mapping of transit depth to planet radius: a planet ``looks bigger'' in a transit depth if it blocks the brighter-on-average flux of the stellar disk.  \textbf{You cannot measure a planet radius without making some assumptions about the extent, contrast, and distribution of starspots.}

The prevailing assumptions within the exoplanet community has been that starspot covering fractions are small enough that the 


\subsection{Stellar surface structures as a gateway for brown dwarf cloud modelin}
more yjumd

\section{Analysis Plan}

\subsection{Get amplitudes and periods for a large fraction of stars from TESS data}
\subsection{Compute the typical ratio of Kepler/K2 amplitudes to TESS amplitudes at given SpT/rotation rate}
\subsection{Interpret the Tess/Kepler ratios as spot contrast differences}
\subsection{Get starspot temperature and coverage fraction for a subsample of these stars from spectroscopy}
\subsection{Compute the ratio of TESS amplitude to total spot coverage}
\subsection{Compute a regression of spot coverage to rotation rate and spectral subtype.}


\section{Technical Feasibility}

\subsection{Measuring periods and amplitudes, which has been demonstrated}
\subsection{There will be enough GKM stars}
\subsection{The TESS precision will be adequate}
\subsection{Experience with the spectral technique on young stars}
\subsection{Availability of archival spectroscopy}

\section{Expected Impact}

\subsection{Probabilistic lookup table for spot temperature and spot coverage}
The probabilistic lookup table for spot temperature and filling factor will empower exoplanet researchers to constrain biases in exoplanet transit radii.

\subsection{Direct estimate of spot coverage for selected high value targets}
The spectroscopic analysis component of this work will add long-term value to the TESS legacy.

\subsection{Trends in spot coverage with lightcurve periodic substructure}
The adopted Fourier analysis technique naturally collates the relative amplitudes of periodic structures, which encode some additional information about the spatial distribution of stellar surface inhomogeneities, and/or differential rotation.  The collated harmonics will enable exploratory analyses of trends with other observables, as recently proposed in the amount of time a spot-modulated \kepler\ lightcurve exhibits single or double dips \citep{2018ApJ...863..190B}.

\section{Work Plan}

The Pincipal Investigator (PI) will carry out the data analysis using custom scripts that build upon the heritage of the Kepler data analysis ecosystem, including \texttt{lightkurve}, \texttt{astropy}, \texttt{celerite}, \texttt{BombScargle}, \texttt{PyTorch}, and \texttt{bokeh}.  The PI will first apply these scripts to Kepler/K2 lightcurves while in their current role as support scientist at the Kepler/K2 Guest Observer Office.  The PI will relocate to the University of Texas at Austin in February 2020 as a Research Fellow.  UT Austin undergraduate and graduate students will lead the application of TESS-adapted scripts to \tess\ FFI lightcurves.  \tess-adapted scripts will leverage \texttt{TESS-cut} and \texttt{elenor} to produce lightcurves from FFIs.  Half of the funding will go to summer 2020 student salaries, and half will go to support the PI.  All reproducible scripts---including Jupyter Notebook tutorials---will be publicly available and developed openly on GitHub.  MGS will provide students with an interactive \texttt{bokeh} tool to visualize the Fourier decomposition of a periodic lightcurve into its constituent components.  This tool will leverage the PI's familiarity with \texttt{bokeh} to aid understanding and rapid visual spotchecking of the multidimensional analysis output.  Finally, a High Level Science Product (HLSP) will be delivered to MAST that contains the following information for each star found to possess detectable starspot-induced lightcurve modulation: 1) the period or periods, 2) an iteratively determined outlier mask of stellar flares, occultations, or spurious data artifacts, 3) a vector of Chebychev polynomial coefficients to account for any long-term normalization trends on timescales comparable to the observation window, and 4) an $N_{\mathrm{Fourier}}$ vector of Fourier harmonic amplitudes that reconstruct the periodic structures in the lightcurve to within the signal-to-noise ratio of the data.  MGS will advise students in the writeup of the paper(s), authored and/or coauthored by students.

\bibliography{tessgi_gully_cycle2}

\end{document}
