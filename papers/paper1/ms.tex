\documentclass[modern]{aastex631}
\bibliographystyle{aasjournal}

%\turnoffedit

% \usepackage{fontspec}
% \usepackage[T1]{fontenc}
% \usepackage{newtxsf}
% \setmainfont{Fira Sans Book}[Scale=1.0]

\usepackage[caption=false]{subfig}
\usepackage{censor}
\usepackage{booktabs}
\usepackage{graphicx}
\usepackage{float}
\usepackage[section]{placeins}
\usepackage{csquotes}
\graphicspath{{./figures/}}

\begin{document}
\shorttitle{Starspot contrast with TESS and K2}
\shortauthors{TBD}
\title{A systematic measurement on starspot contrast from TESS and K2
  bandpass differences}

\author{TBD}
\affiliation{University of Texas at Austin Department of Astronomy}

\author{TBD}
\affiliation{TBD}


\begin{abstract}

  Abstract goes here.

\end{abstract}

\keywords{Starspots (1572)}

\section{Introduction}\label{sec:intro}

Here is an annotated bibliography.

\begin{deluxetable}{chc}
  \tablecaption{Annotated bibliography for intro\label{table1}}
  \tablehead{
    \colhead{Reference} & \nocolhead{two} & \colhead{Key idea}
  }
  \startdata
  \citet{gullysantiago17} & - & LkCa~4 starspot spectral decomposition\\
  \citet{2015ApJ...807..174S} & - & Radius inflation from spots \\
  Strassmeier & - & Starspots review \\
  Rackham & - & TLSE \\
  \enddata
\end{deluxetable}



\section{Observations and Data Reduction}

All sources were observed by both the K2 \citep{howell14} and TESS \citep{2015JATIS...1a4003R} missions.

\subsection{Sample selection}
Our initial sample mirrors...

\subsection{TESS pipelines}


\subsection{Quality assurance checks}

\subsubsection{Handling data gaps}

\subsubsection{Handling multiple sectors}

\subsubsection{Quality flags}

\subsubsection{Outlier detection}

At first, the lightcurve program was falsely calculating many periods due to possible instrumental artifacts, flares, exoplanets, and background noise. To mitigate these outliers, we first tried using the \enquote{remove\_nans} and \enquote{remove\_outliers} methods. This helped, but still left many artifacts. Next, we decided to limit the period calculation to a maximum of 10 days and a minimum of 0.1 days.

After we finished extrapolating the data using the computer, we then developed several criteria by close inspection to further eliminate outliers. One such criterion was to discard any points that exceeded or subceeded 80\% of perfect correlation between the missions. and points with over a 7 days period. The first requirement was decided because anything that differs by more than 80\%, in our opinion, most certainly the result of data fluctuations. Similarly, we have to exclude any periods greater than 7 days due to TESS's time restraints. TESS only collected data for each object for around 22 days, so we set a 7-day maximum period to include 3 whole wavelengths.


\authorcomment1{Outliers attributable to instrumental artifacts}.

\authorcomment1{Outliers attributable to flares}.

\authorcomment2{Outliers attributable to exoplanets}.

\begin{figure}[!htb]
  \centering
  \includegraphics[scale=0.5]{Amplitude Comparison Mask.png}
  \caption{The mask showing how we determined which points were outliers.}
\end{figure}
\FloatBarrier

\section{Analysis}

\subsection{Period measurement}


\subsubsection{Amplitude measurements}


  % Helpful resource on how to insert and align images
  % https://www.overleaf.com/learn/latex/Inserting_Images
  \begin{figure}[!htb]
    \centering
    \includegraphics[scale=0.5]{Comparison Between TESS and Kepler Amplitudes.png}
    \caption{Relationship between the calculated amplitudes of TESS and Kepler and how close they are to unity.}
  \end{figure}
  \FloatBarrier


\section{Results}

\subsection{Amplitude versus rotation period for bins in $T_{\mathrm{eff}}$}

\subsection{TESS-to-K2 amplitude ratio as a function of $T_{\mathrm{eff}}$}

 As seen in the figure below, the Kepler data are, on average, around 70 times larger than TESS's.

  \begin{figure}[!htb]
    \centering
    \includegraphics[scale=0.42]{Amplitude vs. Rotation for Kepler and TESS.png}
    \caption{Effect size for starspot contrast as seen in modulation amplitude versus period ($P_{\mathrm{rot}}$ [days]). The black points are the locus of spotted stars in the Kepler as measured by for late K dwarfs. The red points are the same stars, but measured with the TESS mission. The blue and green dashed lines represents the average amplitudes for all plotted stars for both Kepler and TESS, approximately 9350 and 130, respectively.}
  \end{figure}
  \FloatBarrier

\subsection{Converting amplitude ratio to $T_{\mathrm{spot}}$}

\subsection{Inferred $T_{\mathrm{spot}}$ versus $T_{\mathrm{eff}}$}


\section{Discussion}

\subsection{Comparison to other studies}
\citet{2016MNRAS.463.2494F} targeted 304 Pleiades sources with LAMOST and found the $T_{\mathrm{spot}}$ scales with $T_{\mathrm{eff}}$ with figure.

\subsection{Assessing the spots versus faculae}
Up to the this point, we have assumed that all flux variations arise from dark spots on the surface of the star. Stars can just as likely have bright patches that cause flux perturbations to their light curves. Here we assess the consequences of assuming faculae instead of spots.

\section{Conclusions}

\begin{acknowledgements}
This paper includes data collected by the TESS mission. Funding for the TESS mission is provided by the NASA's Science Mission Directorate.

The authors acknowledge the Texas Advanced Computing Center (TACC, \url{http://www.tacc.utexas.edu}) at The University of Texas at Austin for providing HPC resources that have contributed to the research results reported within this paper.
\end{acknowledgements}

\clearpage


\facilities{Gemini:South, TESS, ASAS, Gaia}

\software{  
  pandas \citep{mckinney10, reback2020pandas},
  emcee \citep{foreman13},
  matplotlib \citep{hunter07},
  astroplan \citep{astroplan2018},
  astropy \citep{exoplanet:astropy13,exoplanet:astropy18},
  exoplanet \citep{exoplanet:exoplanet},
  numpy \citep{harris2020array},
  scipy \citep{jones01},
  ipython \citep{perez07},
  bokeh \citep{bokehcite},
  seaborn \citep{waskom14}}
%pytorch \citep{NEURIPS2019_9015}} % No pytorch yet!


\bibliography{ms}


\clearpage

\appendix
\restartappendixnumbering

\section{Optional appendix} \label{appendix:tools}

Place optional content here.
\end{document}